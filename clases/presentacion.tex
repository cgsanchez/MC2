\documentclass{beamer}

\usepackage[utf8]{inputenc}
\usepackage[spanish]{babel}

\usepackage{hyperref}
\usepackage{graphicx}
\usepackage{amsmath}

\hypersetup{
    colorlinks = true,
    allcolors = blue
}

\usetheme{Madrid}
\usecolortheme{seagull}

%Information to be included in the title page:
\title{Mecánica Cuántica II ($mc_2$)}
\subtitle{Presentación}
\author{Cristián G. Sánchez}
\institute{Facultad de Ciencias Exactas y Naturales - UNCuyo\\
        Instituto Interdisciplinario de Ciencias Básicas - CONICET}


\date{2021}

\begin{document}

\frame{\titlepage}

%%%%%%%%%%%%%%%%%%%%%%%%%%%%%%%%%%%%%%%%%%%%%%%%%%%%%%%%%%
\begin{frame}
\frametitle{Yo (Ió)}
\framesubtitle{Cristián Gabriel Sánchez}

\begin{itemize}
    \item 25 años de experiencia docente.
    \item Químico de formación, Físico por usucapión.
    \item Queen's University Belfast, Universidad Nacional de Córdoba y ahora UNCuyo.
    \item Mi tema de investigación es la Dinámica Cuántica, en particular:
    espectroscopía (no lineal) de fases condensadas y nanosistemas.
    \item Utilizo simulación computacional como principal herramienta.
\end{itemize}

\end{frame}

%%%%%%%%%%%%%%%%%%%%%%%%%%%%%%%%%%%%%%%%%%%%%%%%%%%%%%%%%
\begin{frame}
    \frametitle{El espacio Curricular}
    \framesubtitle{Mecánica Cuántica II}

\begin{block}{Expectativas de logro}
    \begin{itemize}
        \item Aplicar técnicas de {\em \color{blue} teorías de perturbaciones} para encontrar    
            soluciones aproximadas a problemas de {\em \color{blue}  estado estacionario} y     
            {\em \color{blue}  dependientes del tiempo} en mecánica cuántica. \pause
        \item Aplicar {\em \color{blue}  técnicas básicas de segunda cuantización} para plantear   
             estrategias de solución a problemas en sistemas de partículas    
             idénticas en el marco de la mecánica cuántica. \pause
        \item Describir las consecuencias del {\em \color{blue}  fenómeno de entrelazamiento} en   
             el modelo del mundo físico que propone la mecánica cuántica,     
             incluyendo su {\em \color{blue} posible aplicación a la computación}.  
    \end{itemize}
\end{block}
    
\end{frame}

%%%%%%%%%%%%%%%%%%%%%%%%%%%%%%%%%%%%%%%%%%%%%%%%%%%%%%%%%
\begin{frame}
    \frametitle{El espacio Curricular}
    \framesubtitle{Mecánica Cuántica II}

\begin{block}{En síntesis}
    Adquirir una serie de herramientas físicas y matemáticas para la solución
    de problemas físicamente relevantes en mecánica cuántica.
\end{block}
    
\end{frame}

%%%%%%%%%%%%%%%%%%%%%%%%%%%%%%%%%%%%%%%%%%%%%%%%%%%%%%%%%
\begin{frame}
    \frametitle{Mecánica Cuántica II}
    \framesubtitle{Unidad 1}

\begin{block}{Operadores, Matrices, Vectores de Estado, Evolución temporal}
    Los axiomas de la mecánica cuántica. El álgebra de la mecánica       
    cuántica. Notación de Dirac. Bases discretas y continuas. Evolución  
    Temporal. Representaciones de Schrödinger y Heisenber. Matriz        
    densidad. Estados puros y mezclas. Matriz densidad reducida.         
    Evolución temporal de la matriz densidad. Ecuación de Liouville-Von  
    Neumann.    
\end{block}
    
\end{frame}

%%%%%%%%%%%%%%%%%%%%%%%%%%%%%%%%%%%%%%%%%%%%%%%%%%%%%%%%%
\begin{frame}
    \frametitle{Mecánica Cuántica II}
    \framesubtitle{Unidad 2}

\begin{block}{Métodos aproximados para estado estacionarios}
    Teoría de perturbaciones de Rayleigh-Schrödinger: Para sistemas no   
     degenerados, para sistemas degenerados. El Principio Variacional.    
     Teoría de perturbaciones de Brillouin-Wigner. Aplicaciones a         
     sistemas modelo (oscilador armónico, partícula en una caja), física  
     atómica y molecular, etc.. 
\end{block}
    
\end{frame}

%%%%%%%%%%%%%%%%%%%%%%%%%%%%%%%%%%%%%%%%%%%%%%%%%%%%%%%%%
\begin{frame}
    \frametitle{Mecánica Cuántica II}
    \framesubtitle{Unidad 3}

\begin{block}{Métodos aproximados para fenómenos dependientes del tiempo}
    Representación de Interacción. Serie de Dyson para el operador       
     evolución, Aproximación súbita, Teoría de perturbaciones dependiente 
     del tiempo. Perturbación armónica, Regla(s) de Oro de Fermi:         
     discreto a discreto, discreto a continuo, continuo a continuo.       
     Aplicaciones a diversos tipos de espectroscopías.
\end{block}
    
\end{frame}

%%%%%%%%%%%%%%%%%%%%%%%%%%%%%%%%%%%%%%%%%%%%%%%%%%%%%%%%%
\begin{frame}
    \frametitle{Mecánica Cuántica II}
    \framesubtitle{Unidad 4}

\begin{block}{Sistemas de partículas idénticas}
    Partículas idénticas. Estados de muchas partículas, Simetría de      
     permutación, Estados completamente simétricos y antisimétricos.      
     Bosones y Fermiones.   
\end{block}
    
\end{frame}

%%%%%%%%%%%%%%%%%%%%%%%%%%%%%%%%%%%%%%%%%%%%%%%%%%%%%%%%%
\begin{frame}
    \frametitle{Mecánica Cuántica II}
    \framesubtitle{Unidad 5}

\begin{block}{Segunda cuantización}
    Espacio de Fock. Operadores de creación y aniquilación para          
     Fermiones y Bosones, propiedades. Operadores de una y dos partículas 
     en segunda cuantización. Transformación de Base. Segunda             
     cuantización para una base continua: Operadores de campo. Ecuación   
     de movimiento de los operadores de campo en la representación de     
     Heisenberg. Aproximaciones de campo medio. Aplicaciones al gas de    
     electrones y la óptica cuántica. 
\end{block}
    
\end{frame}

%%%%%%%%%%%%%%%%%%%%%%%%%%%%%%%%%%%%%%%%%%%%%%%%%%%%%%%%%
\begin{frame}
    \frametitle{Mecánica Cuántica II}
    \framesubtitle{Unidad 6}

\begin{block}{Cimientos de la mecánica cuántica}
    Modelo de Medición de Von-Neumann, Entrelazamiento, Decoherencia, La 
    paradoja EPR, Desigualdad de Bell, Desigualdad CHSH, Teorema de    
     Kochen-Specker.  
\end{block}
    
\end{frame}

%%%%%%%%%%%%%%%%%%%%%%%%%%%%%%%%%%%%%%%%%%%%%%%%%%%%%%%%%
\begin{frame}
    \frametitle{Mecánica Cuántica II}
    \framesubtitle{Unidad 7}

\begin{block}{Nociones de computación cuántica}
    Qubits, Puertas lógicas cuánticas, Circuitos cuánticos, Conjuntos    
     universales de puertas lógicas cuánticas, Codificación superdensa,   
     Introducción a los algoritmos cuánticos.  
\end{block}
    
\end{frame}

%%%%%%%%%%%%%%%%%%%%%%%%%%%%%%%%%%%%%%%%%%%%%%%%%%%%%%%%%
\begin{frame}
    \frametitle{Metodología de Enseñanza}
    \framesubtitle{$mc_2$}

\begin{block}{Material}
    \begin{itemize}
        \item Videos de clases (como éste).
        \item Notas: ya sea lo utilizado en el teórico u otras complementarias.
        \item Ejercicitación: algunos ejercicios ''a entregar'' y otros no.
        \item Clases de consulta semanales (horario a acordar entre todes)
        \item Bibliografía (hay una general en el programa, la vamos a ir detallando a medida avancemos)
        \item La plataforma a usar será Google Classroom.
    \end{itemize}
\end{block}
    
\end{frame}

%%%%%%%%%%%%%%%%%%%%%%%%%%%%%%%%%%%%%%%%%%%%%%%%%%%%%%%%%
\begin{frame}
    \frametitle{Metodología de Evaluación}
    \framesubtitle{$mc_2$}

\begin{block}{Evaluación}
    \begin{itemize}
        \item Dos parciales (6 para regularizar y 8 para {\bf promocionar})
        \item 70\% de los ejercicios a entregar (condición para regularizar).
        \item Tiempos {\bf \color{red}} flexibles.
    \end{itemize}

\end{block}
    
\end{frame}

%%%%%%%%%%%%%%%%%%%%%%%%%%%%%%%%%%%%%%%%%%%%%%%%%%%%%%%%%
\begin{frame}
    \frametitle{Contacto}
    \framesubtitle{$mc_2$}

Mi email es \href{mailto:cristian.g.sanchez@gmail.com}{\tt cristian.g.sanchez@gmail.com}
\vspace{1cm}

Por favor llenar el formulario de preinscripción en: \href{http://bit.ly/mc2fcen}{\tt http://bit.ly/mc2fcen}
\vspace{1cm}

La información general en el flyer:

% AGREGAR ACÁ
    
\end{frame}


%%%%%%%%%%%%%%%%%%%%%%%%%%%%%%%%%%%%%%%%%%%%%%%%%%%%%%%%%%%
\begin{frame}
\frametitle{Síntesis y recursos:}

\begin{itemize}
    \item \href{https://cgsanchez.net/MC2/pdfs/Programa_QM2_2021.pdf}{Programa oficial}
    \item \href{https://cgsanchez.net/QM2/}{Página web del curso (por ahora vacía)}
    \item \href{https://github.com/cgsanchez/MC2}{Repositorio del material en GitHub}
    \item \href{http://bit.ly/mc2fcen}{Formulario de Preinscripción}
\end{itemize}
\vspace{1cm}
\begin{center}
    Los enlaces están en la descripción del video.
\end{center}
\end{frame}

\end{document}
