\documentclass{beamer}

\usepackage[utf8]{inputenc}
\usepackage[spanish]{babel}

\usepackage{hyperref}
\usepackage{graphicx}
\usepackage{amsmath}

\hypersetup{
    colorlinks = true
}

\usetheme{Madrid}

%Information to be included in the title page:
\title{Mecánica Cuántica II}
\subtitle{Template}
%\author{Cristián G. Sánchez}


\date{2021}

\begin{document}

\frame{\titlepage}

%%%%%%%%%%%%%%%%%%%%%%%%%%%%%%%%%%%%%%%%%%%%%%%%%%%%%%%%%%
\begin{frame}
\frametitle{Título}
\framesubtitle{Subtítulo}

\begin{theorem}
    Este es ÉL teorema
\end{theorem}

\begin{proof}
    Esta es la prueba
\end{proof}

\end{frame}

%%%%%%%%%%%%%%%%%%%%%%%%%%%%%%%%%%%%%%%%%%%%%%%%%%%%%%%%%
\begin{frame}
    \frametitle{Título}
    \framesubtitle{Subtítulo}

\begin{block}{Remark}
    Sample text
    \end{block}
    
    \begin{alertblock}{Important theorem}
        \begin{itemize}
            \item Un ítem.
            \item Otro ítem.
        \end{itemize}
    \end{alertblock}
    
    \begin{examples}
    Sample text in green box. The title {\color{cyan} of the block} is ``Examples".
    \end{examples}
\end{frame}


%%%%%%%%%%%%%%%%%%%%%%%%%%%%%%%%%%%%%%%%%%%%%%%%%%%%%%%%%%%
\begin{frame}
\frametitle{Síntesis y recursos:}

\begin{itemize}
\item \href{https://en.wikipedia.org/wiki/Quantum_mechanics}{Mecánica Cuántica en Wikipedia}
\end{itemize}
\end{frame}

\end{document}
