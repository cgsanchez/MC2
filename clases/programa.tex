\documentclass{article}[a4paper,10pt]

\usepackage[utf8]{inputenc}
\usepackage[spanish]{babel}

\usepackage{hyperref}
\usepackage{graphicx}
\usepackage{amsmath}

\title{Programa de Mecánica Cuántica II}
\author{Cristián G. Sánchez}

\begin{document}

\maketitle

\begin{enumerate}
    \item[ {\bf Unidad 1:}] {\bf Operadores, Matrices, Vectores de Estado, Evolución temporal:\\} Los axiomas de la mecánica cuántica. El álgebra de la mecánica cuántica. Notación de Dirac. Bases discretas y continuas. Evolución  Temporal. Representaciones de Schrödinger y Heisenberg. Matriz densidad. Estados puros y mezclas. Matriz densidad reducida. Evolución temporal de la matriz densidad. Ecuación de Liouville-Von Neumann. 
    \item [{\bf Unidad 2:}] {\bf Métodos aproximados para estado estacionarios:\\} Teoría de perturbaciones de Rayleigh-Schrödinger: Para sistemas no degenerados, para sistemas degenerados. El Principio Variacional. Teoría de perturbaciones de Brillouin-Wigner. Aplicaciones a sistemas modelo (oscilador armónico, partícula en una caja), física  
    atómica y molecular, etc..
    \item [{\bf Unidad 3:}] {\bf Métodos aproximados para fenómenos dependientes del tiempo:\\} Representación de Interacción. Serie de Dyson para el operador evolución, Aproximación súbita, Teoría de perturbaciones dependiente del tiempo. Perturbación armónica, Regla(s) de Oro de Fermi: discreto a discreto, discreto a continuo, continuo a continuo. Aplicaciones a diversos tipos de espectroscopías.
    \item [{\bf Unidad 4:}] {\bf Sistemas de partículas idénticas:\\} Partículas idénticas. Estados de muchas partículas, Simetría de permutación, Estados completamente simétricos y antisimétricos. Bosones y Fermiones.
    \item [{\bf Unidad 5:}] {\bf Segunda cuantización:\\} Espacio de Fock. Operadores de creación y aniquilación para Fermiones y Bosones, propiedades. Operadores de una y dos partículas en segunda cuantización. Transformación de Base. Segunda cuantización para una base continua: Operadores de campo. Ecuación de movimiento de los operadores de campo en la representación de Heisenberg. Aproximaciones de campo medio. Aplicaciones al gas de electrones y la óptica cuántica. 
    \item [{\bf Unidad 6:}] {\bf Cimientos de la mecánica cuántica:\\} Modelo de Medición de Von-Neumann, Entrelazamiento Decoherencia, La paradoja EPR, Desigualdad de Bell, Desigualdad CHSH, Teorema de Kochen-Specker.  
    \item [{\bf Unidad 7:}] {\bf Nociones de computación cuántica:\\} Qubits, Puertas lógicas cuánticas, Circuitos cuánticos, Conjuntos universales de puertas lógicas cuánticas, Codificación superdensa, Introducción a los algoritmos cuánticos. 
\end{enumerate}



\end{document}