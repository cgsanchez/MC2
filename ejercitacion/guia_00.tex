\documentclass[a4paper,12pt]{article}
\usepackage[spanish]{babel}
%\usepackage{avant}
%\renewcommand*\familydefault{\sfdefault}
%\usepackage{helvet}
\usepackage[utf8]{inputenc}
\usepackage{graphicx}
\usepackage{booktabs}
\usepackage{mathtools}
\mathtoolsset{showonlyrefs} 
%\usepackage{amsmath}
%\usepackage{amsfonts}

\usepackage{hyperref}

\hypersetup{
    colorlinks = true,
    allcolors = blue
}

\begin{document}
\title{Mecánica Cuántica II}

\author{Guía X: Cosas que pasan}

\date{\today}

\maketitle

\noindent {\bf IMPORTANTE:} 
\begin{itemize}
    \item Leé {\bf atentamente} las consignas.
    \item Utilizá los recursos que consideres necesarios para resolver los ejercicios. Esto incluye bibliografía, web (!considerá con criterio qué fuentes usás¡), etc..
    \item Subí a Classroom la respuesta, {\bf legible}, con el procedimiento de resolución (aunque esté resumido). Se debe poder comprender cómo llegaste al resultado. Una foto de lo hecho a mano sirve, si lo hacés en \LaTeX\ y subís un PDF {\bf mucho mejor}. Sólo es ``necesario'' subir las de los entregables, pero subir todo ayuda al profe a ayudarte.
    \item Si los ejercicios requieren de soluciones algebráicas respetá las indicaciones de hacerlos ``a mano''. En ese caso cualquier control que quieras usar vale SymPy, Sage, Mathematica, etc.. 
\end{itemize}

\section*{Entregables}

\begin{enumerate}
    \item[\bf Ejercicio 1] Cuantizar la gravedad.  
\end{enumerate}

\section*{No Entregables}

\begin{enumerate}
    \item[\bf Ejercicio 1:] Hacer la mortal para atrás.

\end{enumerate}


\end{document}
