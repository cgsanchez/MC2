\documentclass[a4paper,12pt]{article}
\usepackage[spanish]{babel}
%\usepackage{avant}
%\renewcommand*\familydefault{\sfdefault}
%\usepackage{helvet}
\usepackage[utf8]{inputenc}
\usepackage{graphicx}
\usepackage{booktabs}
\usepackage{mathtools}
\mathtoolsset{showonlyrefs} 
%\usepackage{amsmath}
%\usepackage{amsfonts}

\usepackage{hyperref}

\newcommand{\bra}[1]{\langle #1 |}
\newcommand{\ket}[1]{| #1 \rangle}
\newcommand{\braket}[2]{\langle #1 | #2 \rangle}
\newcommand{\brah}[1]{(#1|}
\newcommand{\ham}{\mathcal{H}}
\newcommand{\ii}{\mathrm{i}}
\newcommand{\tr}{\mathrm{Tr}}

\hypersetup{
    colorlinks = true,
    allcolors = blue
}

\begin{document}
\title{Mecánica Cuántica II}

\author{Guía X: Teoría de perturbaciones }

\date{\today}

\maketitle

\noindent {\bf IMPORTANTE:} 
\begin{itemize}
    \item Leé {\bf atentamente} las consignas.
    \item Utilizá los recursos que consideres necesarios para resolver los ejercicios. Esto incluye bibliografía, web (!considerá con criterio qué fuentes usás¡), etc..
    \item Subí a Classroom la respuesta, {\bf legible}, con el procedimiento de resolución (aunque esté resumido). Se debe poder comprender cómo llegaste al resultado. Una foto de lo hecho a mano sirve, si lo hacés en \LaTeX\ y subís un PDF {\bf mucho mejor}. Sólo es ``necesario'' subir las de los entregables, pero subir todo ayuda al profe a ayudarte.
    \item Si los ejercicios requieren de soluciones algebráicas respetá las indicaciones de hacerlos ``a mano''. En ese caso cualquier control que quieras usar vale SymPy, Sage, Mathematica, etc.. 
\end{itemize}

\section*{Entregables}

\begin{enumerate}
    \item[\bf Ejercicio 1:] Dado un Hamiltoniano de la forma:
    \begin{equation}
    \hat{H}=\hat{H}^0-f\hat{A}
    \end{equation}
    donde $f$ es un escalar real. Utilizando la ecuación de Dyson y la expresión de la matriz densidad en términos de la función de Green, calcular la susceptibilidad como el valor de expectación
    \begin{equation}
    \left.\left\langle  \frac{\partial \hat{A}}{\partial f}\right\rangle\right|_{f=0}
    \end{equation}

    \item[\bf Ejercicio 2:] Para la ``diatómica'' de Hamiltoniano 
    \begin{equation}
        H^0 = \begin{bmatrix}
            0 & \beta \\
            \beta & 0 
            \end{bmatrix}
    \end{equation}
    con $\beta$ real menor a cero y el operador momento dipolar
    \begin{equation}
        \mu = \begin{bmatrix}
            d & 0 \\
            0 & -d 
            \end{bmatrix}
    \end{equation}
    con $d$ real positivo y el Hamiltoniano en presencia de un campo eléctrico externo
    \begin{equation}
    \hat{H}=\hat{H}^0-E\hat{\mu}
   	\end{equation}   
    calcular la polarizabilidad
    \begin{equation}
    \alpha = \left.\left\langle  \frac{\partial \hat{\mu}}{\partial E}\right\rangle\right|_{E=0}
    \end{equation}
\end{enumerate}

\vspace{1cm}
\begin{center}
    $\clubsuit$~$\clubsuit$~$\clubsuit$
  \end{center}

\end{document}
