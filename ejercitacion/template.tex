\documentclass[a4paper,12pt]{article}
\usepackage[spanish]{babel}
%\usepackage{avant}
%\renewcommand*\familydefault{\sfdefault}
%\usepackage{helvet}
\usepackage[utf8]{inputenc}
\usepackage{graphicx}
\usepackage{booktabs}
\usepackage{mathtools}
\mathtoolsset{showonlyrefs} 
%\usepackage{amsmath}
%\usepackage{amsfonts}

\usepackage{hyperref}

\hypersetup{
    colorlinks = true,
    allcolors = blue
}

\begin{document}
\title{Mecánica Cuántica II}

\author{Guía 0: (Entrada en calor matemática)}

\date{\today}

\maketitle

\noindent {\bf IMPORTANTE:} 
\begin{itemize}
    \item Leé {\bf atentamente} las consignas.
    \item Utilizá los recursos que consideres necesarios para resolver los ejercicios. Esto incluye bibliografía, web (!considerá con criterio qué fuentes usás¡), etc..
    \item Subí a Classroom la respuesta, {\bf legible}, con el procedimiento de resolución (aunque esté resumido). Se debe poder comprender cómo llegaste al resultado. Una foto de lo hecho a mano sirve, si lo hacés en \LaTeX\ y subís un PDF {\bf mucho mejor}. Sólo es ``necesario'' subir las de los entregables, pero subir todo ayuda al profe a ayudarte.
    \item Si los ejercicios requieren de soluciones algebráicas respetá las indicaciones de hacerlos ``a mano''. En ese caso cualquier control que quieras usar vale SymPy, Sage, Mathematica, etc.. 
\end{itemize}

\section*{Entregables}

\begin{enumerate}
    \item[\bf Ejercicio -1] Enrrolarse en el aula de Google Classroom.
    \item[\bf Ejercicio 0] Hacer un usuario en Discord y unirse al servidor de MC2.   
\end{enumerate}

\section*{No Entregables}

\noindent ¡Hacer todos a mano!

\begin{enumerate}
    \item[\bf Ejercicio 1:] Calcule los autovalores y autovectores de la matriz $H$ definida
    de la forma
    \begin{equation}
        H = \begin{bmatrix}
            0 & \beta \\
            \beta & 0 
            \end{bmatrix}
    \end{equation}
    \item[\bf Ejercicio 2:] Encuentre la matriz $O$ que diagonaliza $H$, es decir que
    \begin{equation}
        O^\dagger H O = D
    \end{equation}
    donde $D$ es diagonal.
    \item[\bf Ejercicio 3:] Encuentre la matriz 
    \begin{equation}
        U = e^{-i\hbar H}
    \end{equation}
    donde H es la matriz del ejercicio 1 y demuestre que es unitaria.
    \item[\bf Ejercicio 4:] Dadas las matrices de Pauli:
    \begin{equation}
        \sigma_x = \begin{bmatrix}
            0 & 1 \\
            1 & 0 
            \end{bmatrix}\;
        \sigma_y = \begin{bmatrix}
            0 & i \\
            i & 0 
            \end{bmatrix}\;
        \sigma_z = \begin{bmatrix}
            1 & 0 \\
            0 & -1 
            \end{bmatrix},
    \end{equation} 
    demostrar que son Hermíticas y Unitarias.
    \item[\bf Ejercicio 5:] 
    Demostrar que las matrices de Pauli, junto con la matriz identidad:
    \begin{equation}
        I = \begin{bmatrix}
            1 & 0 \\
            0 & 1 
            \end{bmatrix},
    \end{equation}forman una base para el espacio vectorial de matrices Hermíticas de $2\times 2$:
    \begin{equation}
    H_2(a,b,c,d) = \begin{bmatrix}
        a & c+id \\
        c-id & b 
        \end{bmatrix}.
    \end{equation}

\end{enumerate}

Puede ser útil las siguiente referencia:
 \begin{itemize}
     \item Axler, S. J. (1997). Linear Algebra Done Right. New York: Springer. ISBN: 0387982582 (se puede bajar de \href{https://drive.google.com/file/d/1_qZOW2kRbgqRhuUlvlGcM7RA3KjVcvuB/view?usp=sharing}{acá})
 \end{itemize}
o cualquier bibliografía de álgebra lineal que te sea familiar. El libro de computación cuántica que la gente llama ``Mike \& Ike'', y que es parte de la bibliografía del curso:
\begin{itemize}
    \item Nielsen, Michael A., Chuang, Isaac L.. (2010). Quantum Computation and Quantum Information, (10th Anniversary Edition), Cambridge University Press. (se puede bajar de \href{https://drive.google.com/file/d/1qk7L6Roq1WlGWSahW66ZX-rwl_wjTxcL/view?usp=sharing}{acá} )
\end{itemize}
tiene un buen capítulo de repaso de álgebra lineal.
\begin{center}
    $\clubsuit$~$\clubsuit$~$\clubsuit$
  \end{center}

\end{document}
