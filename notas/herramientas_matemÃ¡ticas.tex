\documentclass[a4paper,12pt]{article}
\usepackage[spanish]{babel}
%\usepackage{avant}
%\renewcommand*\familydefault{\sfdefault}
%\usepackage{helvet}
\usepackage[utf8]{inputenc}
\usepackage{graphicx}
\usepackage{booktabs}
\usepackage{mathtools}
\mathtoolsset{showonlyrefs} 
\usepackage{amsmath}
\usepackage{amsfonts}
\usepackage{amssymb}

\newtheorem{teo}{Teorema}
\newtheorem{defi}{Definición}
\newtheorem{prop}{Proposición}

\begin{document}
\title{Herramientas matemáticas para $mc_2$}

\author{Cristián G. Sánchez}

\date{}

\maketitle

{\bf OJO!!! Esto es sólo un borrador!!!! No creo que esté listo para este año de cursado}

En estas notas recopilo una serie de definiciones, teoremas, etc. que se utilizan en el desarrollo del curso junto con referencias donde profundizar estos conceptos. La profundización {\em no es necesaria} en el marco del curso pero marca un camino para aquelles que deseen  tomar los muchísimos pasadizos y callejuelas que se abren en cada párrafo de física que une estudie.

\section{Conceptos de Álgebra}

Los conceptos de álgebra, salvo indicación en contrario, son extraídos {\em verbatim ac litteratim} de la referencia \ref{LADR}.

\begin{defi}
    Un {\bf campo} es 
\end{defi}

\begin{defi}
    Un {\bf espacio vectorial} sobre el campo $\mathbb{F}$ es un conjunto $V$ sobre el que se definen una adición y multiplicación (cerradas en $V$) con las siguientes propiedades:
    \begin{itemize}
        \item {\bf conmutatividad}: $u+v=v+u$ para todo par $u,v\in V$
        \item {\bf asociatividad}: $(u+v)+w=u+(v+w)$ y $(ab)v=a(bv)$ para todo $u,v,w\in V$ y $a,b\in\mathbb{F}$
        \item {\bf identidad aditiva}: existe un elemento $0\in V$ tal que  $0+v=v$ para todo $v\in V$
        \item {\bf inversa aditiva}: para todo $v\in V$ existe un $w\in V$ tal que $v+w=0$. 
        \item {\bf identidad multiplicativa}: $1v=v$ para todo $v\in V$
        \item {\bf propiedades distributivas}: $a(u+v)=au+av$ y $(a+b)v=av+bv$ para todo $v, u\in V$ y $a,b\in \mathbb{F}$
    \end{itemize}
\end{defi}

\begin{teo}
    Un espacio vectorial tiene una única identidad aditiva (el vector $0$).
\end{teo}

\begin{teo}
    Cada elemento en un espacio vectorial tiene una única inversa aditiva $(-v)=(-1)v$.
\end{teo}

\begin{defi} Una {\bf combinación lineal} de una lista de vectores $v_1,\ldots,v_m$ en un espacio vectorial $V$ es un vector de la forma
    \[\sum_{i=i}^{m}a_i v_i,\]
donde $a_1,\ldots,a_m\in V$.
\end{defi}

\begin{defi}
    El conjunto de todas las combinaciones lineales de una lista de vectores $v_1,\ldots,v_m$ se llama el {\bf \em span} de $v_1,\ldots,v_m$ y se denota $\mathrm{span}(v_1,\ldots,v_m)$
\end{defi}

\begin{defi}
    Una lista de vectores $v_1,\ldots,v_m$ es {\bf linealmente independiente} si la única eleccción de $a_1,\ldots,a_m$ que hace $\sum_{i=i}^{m}a_i v_i=0$ es $a_1 =\ldots =\ a_m = 0$
\end{defi}

\begin{defi} Una {\bf base} de un espacio vectorial $V$ es una lista de vectores $e_1,\ldots,e_n$, linealmente independientes tales que 
    \[V=\mathrm{span}(e_1,\ldots,e_n)\] 
\end{defi}

\begin{teo}
    Cualquier par de bases de un espacio vectorial de dimensión finita tiene el mismo número de elementos.
\end{teo}

\begin{defi}
    La {\bf dimensión} de un espacio vectorial es el número de elementos en cualquier base para el espacio.
\end{defi}

\begin{defi}

\end{defi}

\begin{defi}
    El {\bf producto escalar} es 
\end{defi}

\begin{defi}
    Una  {\bf norma} es 
\end{defi}

\begin{prop}
    Propiedades de la norma
\end{prop}

\begin{defi}
    Un {\bf espacio de Hilbert}
\end{defi}





\section{Conceptos de Números complejos}

\begin{defi} Números complejos

    \begin{itemize}
        \item Un {\bf número complejo} es un par ordenado $(a,b)$ donde $a,b \in \mathbb{R}$ que se denota $a+ib$.
        \item El conjunto de todos los números complejos se denota $\mathbb{C}$:
        \[ \mathbb{C} = \{a+bi : a,b\in \mathbb{R}\}\]
        \item La {\bf adición y multiplicación} en $\mathbb{C}$ se definen de la forma:
        \[(a+bi) + (c+di) = (a+c)+(b+d)i,\]
        \[(a+bi)(c+di) = (ac-bd)+(ad+bc)i;\]
        donde $a,b,c,d\in\mathbb{R}$
    \end{itemize}

\end{defi}

\begin{teo} Fórmula de Euler

    Para todo $x\in\mathbb{R}$:
    \[e^{ix} = \cos x + i \sin x.\]

\end{teo}

\section{Conceptos de Análisis}

\begin{teo}
    Teorema integral de Cauchy
\end{teo}

\begin{defi}
    Función analítica
\end{defi}



\end{document}